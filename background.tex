\section{Background}
\label{sec:background}

\ld{I still need to clean this up.... Maybe some of the examples could be useful in Approach.}


\paragraph{\textbf{Entity Salience.}} 
Consider the two passages below from two news articles about the entity \textit{Boris Johnson} which address his role as the \textit{Prime Minister of the UK} and his response to the recent COVID-19 pandemic.
\begin{quote}
\textbf{Passage 1.} The British government came under heightened pressure to disclose details about a secretive scientific advisory group after a report on Friday that a top political aide to Prime Minister Boris Johnson had taken part in the group’s meetings on the coronavirus pandemic. \footnote{https://www.nytimes.com/2020/04/25/world/europe/uk-dominic-cummings-sage-coronavirus.html}\\
\textbf{Passage 2.} British Prime Minister Boris Johnson is resisting growing calls to reopen the UK from its lockdown because he is still so “frightened” from his own near-fatal brush with the bug, according to a report \footnote{https://nypost.com/2020/04/21/boris-johnson-too-frightened-to-ease-uk-coronavirus-lockdown/}.
\end{quote}
We notice that Passage 2 discusses how the entity \textit{Boris Johnson} in his role as the Prime Minister of the UK is affecting the pandemic situation, whereas Passage 1 just mentions the entity on the side. The entity is central to the discussion in Passage 2 whereas in Passage 1, it is not. We say that \textit{Boris Johnson} is \textit{salient} in Passage 2. Hence, by \textit{salient}, we mean that the entity is \textit{central} to the text in which it is mentioned. 

Consider the following sentence from a new article about \textit{Boris Johnson}.

\begin{quote}
    Boris Johnson, perhaps the world's most famous coronavirus patient, was back at work Monday — after spending the worst of Britain's epidemic sidelined,
first in self-isolation, then struggling to breathe in the hospital, and later in recovery in the countryside \footnote{https://www.washingtonpost.com/world/europe/boris-johnson-returns-to-work-after-missing-worst-of-coronavirus-epidemic/2020/04/27/95b590ea-8630-11ea-81a3-9690c9881111_story.html}.
\end{quote}

In a sentence such as the one above, we would not only prefer to link the mention \textit{Boris Johnson} to the aspect \textit{Prime Minister of the UK}, but also to one whose content is a passage like Passage 2 and not Passage 1. We hypothesize that entity salience is a useful indicator of aspects for entities. We use SWAT \cite{swat}  to find the salience of an entity in text. Given some text, SWAT outputs the entities along with their salience scores in the text. For example, using the online demo \footnote{https://swat.d4science.org/}, given the two passages above, SWAT correctly predicts \textit{Boris Johnson} as salient in Passage 2 (Score = 0.6), and non-salient in Passage 1 (Score = 0.15). 

\paragraph{\textbf{Entity Relatedness. }}
Entity Relatedness is a measure of how strongly related two entities are. For example, consider the entities, \textit{Boris Johnson}, \textit{Theresa May}, and \textit{Donald Trump}. Intuitively, one would say that \textit{Boris Johnson} is more strongly related to \textit{Theresa May} than to \textit{Donald Trump} because both \textit{Boris Johnson} and \textit{Theresa May} are British politicians and had some role in Brexit. We hypothesize that this measure of entity relatedness can help in aspect linking. More concretely, we hypothesize that an aspect mentioning many related entities to the target entity (the entity we are trying to aspect link), is a good candidate for an aspect for the given target entity. 
We use the Entity Relatedness system from WAT \cite{piccinno2014wat} to find relatedness between pairs of entities. Given a list of entities, WAT provides the
relatedness measure between every pair of entities in the list. For example, given the entity list consisting of \textit{Boris Johnson}, \textit{Theresa May} and \textit{Donald Trump}, WAT predicts the relatedness between every pair of entities as follows:
\begin{quote}
    (\text{Boris Johnson}, \text{Donald Trump}) = 0.37, \\
    (\text{Theresa May},\text{Donald Trump})    = 0.38, \\
    (\text{Boris Johnson}, \text{Theresa May})  = 0.67
\end{quote}

\paragraph{\textbf{Co-occurring Entities.}}
Co-occurring entities are entities which co-occur with a given entity in a particular context such as a sentence, passage or article. For example, entities which might co-occur with \textit{Boris Johnson} in a passage may be \textit{Theresa May}, \textit{United Kingdom} and \textit{Brexit}. We study the role of such co-occurring entities on the aspect linking task by comparing whether the frequency or relatedness of these co-occurring entities to the target entity is a better indicator of aspects and under what conditions they work. 
