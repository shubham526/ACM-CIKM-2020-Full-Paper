%
% The first command in your LaTeX source must be the \documentclass command.
\documentclass[sigconf,authordraft]{acmart}

%
% defining the \BibTeX command - from Oren Patashnik's original BibTeX documentation.
\def\BibTeX{{\rm B\kern-.05em{\sc i\kern-.025em b}\kern-.08emT\kern-.1667em\lower.7ex\hbox{E}\kern-.125emX}}
    
% Rights management information. 
% This information is sent to you when you complete the rights form.
% These commands have SAMPLE values in them; it is your responsibility as an author to replace
% the commands and values with those provided to you when you complete the rights form.
%
% These commands are for a PROCEEDINGS abstract or paper.
\copyrightyear{2018}
\acmYear{2018}
\setcopyright{acmlicensed}
\acmConference[Woodstock '18]{Woodstock '18: ACM Symposium on Neural Gaze Detection}{June 03--05, 2018}{Woodstock, NY}
\acmBooktitle{Woodstock '18: ACM Symposium on Neural Gaze Detection, June 03--05, 2018, Woodstock, NY}
\acmPrice{15.00}
\acmDOI{10.1145/1122445.1122456}
\acmISBN{978-1-4503-9999-9/18/06}

%
% These commands are for a JOURNAL article.
%\setcopyright{acmcopyright}
%\acmJournal{TOG}
%\acmYear{2018}\acmVolume{37}\acmNumber{4}\acmArticle{111}\acmMonth{8}
%\acmDOI{10.1145/1122445.1122456}

%
% Submission ID. 
% Use this when submitting an article to a sponsored event. You'll receive a unique submission ID from the organizers
% of the event, and this ID should be used as the parameter to this command.
%\acmSubmissionID{123-A56-BU3}

%
% The majority of ACM publications use numbered citations and references. If you are preparing content for an event
% sponsored by ACM SIGGRAPH, you must use the "author year" style of citations and references. Uncommenting
% the next command will enable that style.
%\citestyle{acmauthoryear}

%
% end of the preamble, start of the body of the document source.
\begin{document}

%
% The "title" command has an optional parameter, allowing the author to define a "short title" to be used in page headers.
\title{Entity Aspect Linking using Entity Salience and Relatedness}

%
% The "author" command and its associated commands are used to define the authors and their affiliations.
% Of note is the shared affiliation of the first two authors, and the "authornote" and "authornotemark" commands
% used to denote shared contribution to the research.
\author{Shubham Chatterjee}
\email{sc1242@cs.unh.edu}
%\orcid{1234-5678-9012}
\affiliation{
 \institution{University of New Hampshire}
  \city{Durham}
 \state{New Hampshire}
}
\author{Jordan Ramsdell}
\email{jsc57@cs.unh.edu}
%\orcid{1234-5678-9012}
\affiliation{
 \institution{University of New Hampshire}
  \city{Durham}
 \state{New Hampshire}
}
\author{Laura Dietz}
\email{dietz@cs.unh.edu}
\affiliation{
 \institution{University of New Hampshire}
  \city{Durham}
 \state{New Hampshire}
}

%
% By default, the full list of authors will be used in the page headers. Often, this list is too long, and will overlap
% other information printed in the page headers. This command allows the author to define a more concise list
% of authors' names for this purpose.
\renewcommand{\shortauthors}{Chatterjee et al.}

%
% The abstract is a short summary of the work to be presented in the article.
\begin{abstract}
Entity linking tools at present provide only coarse-grained information with no knowledge about the different events, topics, roles or in general, aspects of the entity that the mention in the text links to. Entity Aspect Linking is the task of associating with an entity link, the correct aspect of the entity mention in the text. Previous work \cite{nanni2018entity} has shown that a supervised combination of various text and entity features based on embeddings of the words and entities from various sources (context of mention, content of Wikipedia page of the mention, etc) performs very well and is able to correctly predict aspects in 70\% of the cases. In this work, we hypothesize that considering the salience of an entity and its relatedness to other entities may further improve results. Our experiments show that this is indeed the case. We perform extensive experiments using entity salience and entity relatedness and obtain new state-of-the-art results on the task.
\end{abstract}

%
% This command processes the author and affiliation and title information and builds
% the first part of the formatted document.
\maketitle

\section{Introduction}
\label{sec:Introduction}

\section{Related Work}
\label{sec:Related Work}

\section{Approach}
\label{sec:Approach}

\section{Evaluation}
\label{sec:Evaluation}

\section{Conclusion}
\label{sec:Conclusion}

\bibliographystyle{ACM-Reference-Format}
\bibliography{references}
\end{document}
