\section{Approach}
\label{sec:Approach}
Our goal is to enrich entity links with fine-grained aspects of the entities. In this work, we consider whether a static measure of entity salience and relatedness which has not been trained for this task can help. We refer to the entity that we are trying to aspect link as the target entity.

%To study the role of entity salience, entity relatedness and co-occurring entities, we describe below several ranking strategies based on these and later combine them in a learning-to-rank system. In all our discussions, we refer to the entity that we are trying to aspect link as the target entity.

\subsection{Reproduction of Nani Et Al.}

We reimplemented Nanni et al's features as described in background. We explore three machine methods for combining these features, Learning to Rank, Logisitc Network

Additional we further include a BiLSTM feature.


% \subsection{Neural Methods.}
% We use the PyTorch library \cite{pytorch} to construct our neural networks. There are three neural networks that we consider both independently, and together in a joint model. 

\subsection{Entity Aspect Relevance}

\textbf{Logistic Network. }The logistic network consists of an input layer, which takes the entity aspect ``bag of words'' vectors as an input, a single linear layer, and an output layer that represents the likelihood that the entity aspect is the correct aspect to link to a given entity mention. There are three separate versions of this network that we utilize throughout this paper. 


\textbf{BiLSTM Network. } 
ELMo is a \cite{elmo} is a contextual language embedding model that takes as input a sentence, and outputs word embeddings that depend upon the context given by the sentence. Each word embedding consists of three vectors, and in this paper we concatenate these vectors together to obtain a final word embedding. We then average all of the word embeddings in a sentence together to obtain the embedding of a sentence. These embeddings are used as sequential inputs in a BiLSTM. We embed both the paragraph context of an entity's mention, and the text of each candidate entity aspect. (number of hidden layers to be determined ...)
The sentence embeddings of both the entity aspects, and the paragraph context in which an entity mention is located, are used as inputs into two separate BiLSTM layers. The outputs of the BiLSTMs are concatenated and a final linear transformation is used to obtain a scalar value that also represents the likelihood of an entity aspect being the correct aspect to link to a given mention.
 
...
 
 (T.B.D. if BiLSTM approach works with new dataset; may drop if numbers aren't good)
 


\subsection{Entity Salience for Aspect Linking}
\label{subsec:Entity Salience for Aspect Linking}

Consider the two passages in Figure \ref{fig:Salience} from two news articles about the entity \textit{Boris Johnson} which address his role as the \textit{Prime Minister of the UK} and his response to the recent COVID-19 pandemic.
\begin{figure}[t]
    \centering
   \begin{quote}
\textbf{Passage 1.} The British government came under heightened pressure to disclose details about a secretive scientific advisory group after a report on Friday that a top political aide to Prime Minister Boris Johnson had taken part in the group’s meetings on the coronavirus pandemic. \\
\textbf{Passage 2.} British Prime Minister Boris Johnson is resisting growing calls to reopen the UK from its lockdown because he is still so “frightened” from his own near-fatal brush with the bug, according to a report .
\end{quote}
    \caption{Salient versus Non-Salient Passage. Passage 2 is salient for entity \textit{Boris Johnson} whereas Passage 1 is not.}
    \label{fig:Salience}
\end{figure}
We notice that Passage 2 \footnote{https://nypost.com/2020/04/21/boris-johnson-too-frightened-to-ease-uk-coronavirus-lockdown/} in Figure \ref{fig:Salience} discusses how the entity \textit{Boris Johnson} in his role as the Prime Minister of the UK is affecting the pandemic situation, whereas Passage 1 \footnote{https://www.nytimes.com/2020/04/25/world/europe/uk-dominic-cummings-sage-coronavirus.html} just mentions the entity on the side. Hence, the entity is central (\textit{salient}) in Passage 2 whereas in Passage 1, it is not. 
%We say that \textit{Boris Johnson} is \textit{salient} in Passage 2. Hence, by \textit{salient}, we mean that the entity is \textit{central} to the text in which it is mentioned. 
In a sentence such as \textit{Boris Johnson, perhaps the world's most famous coronavirus patient, was back at work on Monday \footnote{https://www.washingtonpost.com/world/europe/boris-johnson-returns-to-work-after-missing-worst-of-coronavirus-epidemic/2020/04/27/95b590ea-8630-11ea-81a3-9690c9881111\_story.html}}, we would not only prefer to link the mention \textit{Boris Johnson} to the aspect \textit{Prime Minister of the UK}, but also to one in which the entity is salient, such as one with content Passage 2.

%Consider the following sentence from a new article about \textit{Boris Johnson}.

%\begin{quote}
%    \textit{Boris Johnson, perhaps the world's most famous coronavirus patient, was back at work Monday — after spending the worst of Britain's epidemic sidelined, first in self-isolation, then struggling to breathe in the hospital, and later in recovery in the countryside \footnote{https://www.washingtonpost.com/world/europe/boris-johnson-returns-to-work-after-missing-worst-of-coronavirus-epidemic/2020/04/27/95b590ea-8630-11ea-81a3-9690c9881111\_story.html}}.
%\end{quote}

%In a sentence such as the one above, we would not only prefer to link the mention \textit{Boris Johnson} to the aspect \textit{Prime Minister of the UK}, but also to one in which the entity is salient. 
We hypothesize that entity salience is a useful indicator of aspects for entities. We use SWAT \cite{swat}  to find the salience of an entity in text. Given some text, SWAT outputs the entities along with their salience scores in the text. For example, using the online demo \footnote{https://swat.d4science.org/}, given the two passages in Figure \ref{fig:Salience}, SWAT correctly predicts \textit{Boris Johnson} as salient in Passage 2 (Score = 0.6), and non-salient in Passage 1 (Score = 0.15). 

\begin{enumerate}
    \item \textbf{Salience of Entity Mention in aspect (Sal-EM).} Score of the aspect is equal to the salience score of the entity mention in the aspect if the entity is salient, and zero otherwise.
    
    \item \textbf{Salient Entities in Context (SEC).} Score an aspect by summing over the salience score of entities $e \in E$, where $E = E_A \cap E_C$, $E_A =$ aspect entities, and $E_C =$ salient entities in sentence, paragraph and section context.
    
    \item \textbf{All Entities in Context (AEC).} 
    To investigate the extent to which salient entities can affect the performance, we also experiment with using both salient and non-salient entities from SWAT for $E_C$ in \ref{subsubsec:Entity salience based features}(2) above. 
\end{enumerate}


\subsection{Entity Relatedness for Aspect Linking}
\label{subsec:Entity Relatedness for Aspect Linking}

Entity Relatedness is a measure of how strongly related two entities are. For example, consider the entities, \textit{Boris Johnson}, \textit{Theresa May}, and \textit{Donald Trump}. Intuitively, one would say that \textit{Boris Johnson} is more strongly related to \textit{Theresa May} than to \textit{Donald Trump} because both \textit{Boris Johnson} and \textit{Theresa May} are British politicians and had some role in Brexit. We hypothesize that this measure of entity relatedness can help in aspect linking. More concretely, we hypothesize that an aspect mentioning many related entities to the target entity  is a good candidate for an aspect for the given target entity. For example, consider the aspect about \textit{Boris Johnson} pertaining to his role as the \textit{Prime Minister of UK} during the COVID-19 pandemic.

\begin{quote}
    \textbf{Aspect.} Prime Minister of UK \\
    \textbf{Content.}
    Former Prime Minister Theresa May has criticised world leaders for failing "to forge a coherent international response" to the coronavirus pandemic. Mrs May's intervention comes as Boris Johnson and Sir Keir Starmer face each other at Prime Minister's Questions for the first time later. \footnote{https://www.bbc.com/news/uk-politics-52553237}.
\end{quote}
The aspect above may be linked to the entity \textit{Boris Johnson} since it mentions several related entities such as \textit{Theresa May} and \textit{Sir Keir Starmer}. 

Co-occurring entities are entities which co-occur with a given entity in a particular context such as a sentence, passage or article. For example, entities which might co-occur with \textit{Boris Johnson} in a passage may be \textit{Theresa May}, \textit{United Kingdom} and \textit{Brexit}. We study the role of such co-occurring entities on the aspect linking task by comparing whether the frequency or relatedness of these co-occurring entities to the target entity is a better indicator of aspects and under what conditions they work. 


\subsubsection{Co-occurring entities with an entity from context}
\label{subsubsec:Co-occurring entities with an entity from context}

For every entity $e$ in the sentence, paragraph and section context around the target entity $e_T$, we derive a distribution over co-occurring entities $e'$
with $e$ using the frequency of co-occurrence and the relatedness of co-occurring entities to $e$. We may treat this distribution as the conditional probability distribution $P(e' \vert e)$. To find the co-occurring entities, we use the Entity Context Document (ECD) \cite{chatterjee2019why} for $e$. The ECD is created by first retrieving a candidate set of passages using $e$ as the query and then concatenating all passages which mention $e$.
We score an aspect $A$ in following ways.

\begin{enumerate}
    \item \textbf{Simple Frequency Distribution (SF-Dist).}
    
    We score aspects using frequently co-occurring entities with an entity $e$ from the sentence, paragraph and section context around the target entity, which also occur in the aspect content. Formally,:
    \begin{equation}
    \label{eq:score-aspect-using-simple-freq-dist}
        \text{Score(A)} = \sum_{e' \in A}P(e' \vert e)
    \end{equation}
    where $e'$ is an entity in the aspect $A$ which also frequently co-occurs with $e$, $e$ is an entity from the context around the target entity, and $P( e' \vert e)$ is the frequency distribution over co-occurring entities.
    
    \item \textbf{Weighted Frequency Distribution (WF-Dist).} 
    
    Our intuition is that co-occurring entities which are more related to the target entity should contribute more to the final aspect scores. To investigate how relatedness of co-occurring entities affects the aspect scores, we experiment with scoring an aspect $A$ using a weighted sum of frequencies of entities $e'$ in the aspect, weighted by the relatedness of $e'$ to the target entity $e_T$. Formally,
    \begin{equation}
        \label{eq:score-aspect-using-weighted-freq-dist}
        \text{Score(A)} = \sum_{e' \in A} \text{Relatedness}(e', e_T) \times P(e' \vert e)
    \end{equation}
    where $e'$ is an entity in the aspect $A$ which also frequently co-occurs with $e$, $e_T$ is the target entity, $e$ is an entity from the context around the target entity $e_T$, and $\text{Relatedness}(e', e_T)$ is found using WAT \cite{piccinno2014wat}.

\item \textbf{Relatedness Distribution (Rel-Dist).} 
Same as \ref{subsubsec:Co-occurring entities with an entity from context}(1), but use relatedness of $e'$ to $e$ to find $P( e' \vert e)$.

\end{enumerate}

\subsubsection{Co-occurring entities with the target entity} 
\label{subsubsec:Co-occurring entities with the target entity}

We use two sources of co-occurring entities with the target entity $e_T$: Entity Context Document (\textbf{ECD}) for $e_T$ and the Wikipedia page (\textbf{Wiki}) of $e_T$. We then score an aspect using the co-occurring entities with $e_T$ in two ways.

    
\begin{enumerate}
    \item Use Equation \ref{eq:score-aspect-using-simple-freq-dist} with either frequency distribution \textbf{(SF-Dist-ECD)} or relatedness distribution \textbf{(Rel-Dist-ECD/Rel-Dist-Wiki)} over co-occurring entities to score an aspect.
        
    \item Retrieval Score of Aspect (\textbf{RS-Asp}). We treat the target entity $e_T$ as the query and expand the query using top-20 (according to frequency (\textbf{Freq}) and relatedness (\textbf{Rel})) entities from the ECD/Wikipedia page for $e_T$. We then retrieve aspects from an aspect index for $e_T$ (\textbf{RS-Asp-Freq-ECD/RS-Asp-Rel-ECD/RS-Asp-Rel-Wiki}). We experiment with the combinations of the following retrieval models: BM25 (default Lucene), Language Models with Jelinek-Mercer (LMJM) smoothing $(\lambda = 0.4)$, and Language Models with Dirichlet (LMDS) smoothing, and the following expansion models: RM1 and RM3.
\end{enumerate}
    

\subsubsection{Lexical and Semantic features}
\label{subsubsec:Lexical and Semantic features} 
We re-implemented the lexical and semantic features from Nanni et
al.\cite{nanni2018entity}. To give a brief recap, Nanni et al.\cite{nanni2018entity} consider three types of aspect representations and rank aspects based on similarity of mention in context to:(1) Header, (2) Content and, (3) Entity overlap with each section on the Wikipedia page of the entity. They use four methods to derive features from the aspect representations above: (1) TF-IDF, (2) BM25, (3) Word Embeddings and, (4) Entity Embeddings.

\subsection{Entity Aspect Representation}\label{jordan-vector}
In this paper we represent entity aspects in two ways.
The first way is through the use of features which measure the relevance of an entity aspect with respect to the paragraph context of an entity mention. We use the following following aspect relevance features:

\textbf{bm25-\{sentence / paragraph\}-to-text}. We use the BM25 to compare an entity aspect's main body of text to the sentence, or paragraph, that contains the entity mention. 

\textbf{bm25-\{sentence / paragraph\}-to-header}. We use the BM25 to compare an entity aspect's section header to the sentence, or paragraph, that contains the entity mention. 


\textbf{entity-overlap-\{sentence / paragraph\}}. We count the number of distinct entity overlaps between an entity aspect and the entities contained in the sentence, or paragraph, that contains the entity mention.

\textbf{word2vec-\{sentence / paragraph\}-cosine}. We use the Word2Vec embedding to measure the cosine similarity between the text of an entity aspect, and the text of the sentence, or paragraph, that contains the entity mention.

\textbf{aspect-size}. This length of an entity aspect text in words.

... 

(\textit{actually maybe I should just cite Fede's features and say I used those})

...



 \subsection{Joint Aspect Linking}\label{jordan-co-entity}
One valuable source of context for which we can use to link an aspect to an entity mention, is that entities can be mentioned together in the same paragraph.
We can use this context to further evaluate the relevance of aspects, given how much they have in common with the aspects of entities that co-occur in the same passage. 

For example, the Chocolate Wikipedia page contains many sections that cover a broad array of subtopics. One of these sections is the ``Health Benefits of Chocolate'', which is notably smaller than the other sections in comparison. The section contains very specific medical terms that may not occur in a sentence where this section would be the correct aspect to link to. As an example, consider a sentence like ``Make sure that you keep away chocolate from your dogs.''. There is no indication that this has anything to do with health, but if we first linked the aspect of dogs, we would see that the ``Health'' section explicitly mentions the danger of poisoning from chocolate. Using this context, it then becomes clear what the relevant section is in Chocolate: not the section that mentions chocolate the most, but the one that talks about Theobromine poisoning and side effects.

\textbf{Aspect Relevance.} Our task requires that we choose the aspect that best explains the context of an entity mention. We use a measure of relevance to represent the likelihood that an entity aspect should be linked to an entity given a mention and its surrounding context.

\textbf{Co-Entity Similarity}. It is not always the case that we can use the context of co-entities to infer the context of an entity mention. For example, a very large paragraph can contain many co-entities that are spaced far apart. We would expect distance entities to share little context in common.
We use measures of similarity between co-entities and entities to determine how much context they share in common.

\textbf{Co-Aspect Similarity.}. One drawback of the Co-Entity similarity measure is that it is more of a global measure of contextual similarity. Consider the ``Chocolate'' example in section \ref{jordan-disambig}. Dogs and chocolates are not very related to each other, although we saw a context in which these two entities were related to each other, and that's the fact that chocolate is poisonous to dogs. Both entities have a section discussing this fact, but these sections are far smaller than the other sections in these entities by comparison. In the case of a global measure of inference, we would expect there to be a very weak correlation between chocolate and dogs, as they possess few overlapping topics. If we instead looked at the individual sections in each page, we would see that the ``Health'' section in ``Dog'' frequently mentions the dangers of chocolate and theobromine poisoning. 
 
The global context network utilizes entities that co-occur within the same text as the primary entity mentions. We do not possess a ground truth for the majority of these co-occurring entities because most links in Wikipedia do not link to a section in a page. However, these co-occurring entities are still useful in determining the likelihood that a primary entity aspect aspect should be linked. 
We accomplish this by scoring each pair of co-occurring entity aspect and primary entity aspect with respect to a set of aspect-to-aspect similarity functions. Each similarity function therefore generates an I by J by K matrix, where I is the number of co-occurring entities, J is the number of co-occurring entity aspects which is padded to maintain a size of 20, and K is the number of primary entity aspects. We use the following aspect-to-aspect similarity functions:

\subsubsection{Aspect-to-Aspect Similarities}


\textbf{Unigram Overlap.} The number of unigrams that overlap based on the text contained in each entity aspect.


\textbf{Character n-gram Overlap.} We use a sliding window to generate character 4-grams from each entity aspect text, and the similarity between to entity aspects is respect to the overlap of these 4-grams.

\textbf{Header Overlap.}
We also use the Character n-gram Overlap method with respect to the section header of each entity aspect.

\textbf{Entity Overlap.}
Each entity aspect also contains a number of entity links within the main body of its text. We count the overlap of these entities with respect to two entity aspects.

\textbf{Word2Vec Cosine Similarity.} 
We embed the text of each entity aspect by using Word2Vec to generate word embeddings, and then we average these embeddings together. We measure the cosine similarity between two entities with respect to these embeddings.

We then score the relevance of an entity aspect using a weighted combination of these aspect-to-aspect similarity features.  
Let $a^{'}$ be an entity aspect, $E$ be the collection of co-entities for a given paragraph context, $A_i$ be the set of co-entity aspects with respect to a given co-entity, $e_i$, $S$ be the collection of aspect-to-aspect similarity features, and $\lambda_j$ be a weight for aspect-to-aspect similarity feature $\lambda_j$. Finally, we also utilize a logistic layer to represent the relevance of a co-aspect with respect to a co-entity, given by the function $\text{rel}(e_i, a_j)$. 

\begin{equation}
\text{global}(a^{'}) = \sum_{e_i  \in E}\sum_{a_j \in A_i}\sum_{\text{sim}_k \in S} \text{rel}(e_i, a_j) \cdot \text{sim}_k(a^{'}, a_j) \cdot \lambda_j
\end{equation}

\subsection{Joint Learning}
...(T.B.D. on what works; two different approaches)




\ld{End of Jordan's stuff...}



\begin{comment}
We give a brief overview of the same again for reference. 

Nanni et al.\cite{nanni2018entity} consider three types of aspect representations for finding its similarity to the entity mention in context. 

\textbf{Header.} Rank aspects based on similarity of the mention in context to the header of each section in the Wikipedia page.

\textbf{Content.} Rank aspects based on the similarity between the mention in context and the content of each section of the Wikipedia page of the entity.

\textbf{Entity.} Overlap of entities mentioned in the context of the entity mention and the content of a section on the Wikipedia page of the entity.

They use the following features to rank aspects using the representations above.

\textbf{TF-IDF.} Cosine similarity between the TF-IDF (logarithmic, L2-normalized) vector of contextual mention and aspect.

\textbf{BM25.} Rank aspect representations using the contextual mention as a query using BM25 ($k_1=2, b=0.75)$.

\textbf{Word Embeddings.} Cosine similarity between the mention in context and
the aspect using pre-trained GloVe \cite{pennington2014glove} embeddings of dimension 300. 

\textbf{Entity Embeddings.} Use 500 dimensional RDF2Vec \cite{ristoski2016rdf2vec} embeddings to embed entities in the context of the entity mention and a section from the Wikipedia page of the entity, then compute
the document vector using the TF-IDF of an an entity in context of the entity mention and its embedding.
\end{comment}