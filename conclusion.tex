\section{Conclusion}
\label{sec:Conclusion}
This work addresses the task of entity aspect linking and studies the effectiveness of using entity salience and relatedness on the task using two off-the-shelf tools not trained for this task. We show that although these tools are not perfect and do not pose a solution on their own, a supervised combination of salience and relatedness features with lexical and semantic features can outperform several established baselines. In particular, we show that such a supervised combination learns complementary information which aids the performance of the supervised system. Moreover, we find that using the frequency of co-occurring entities is better than using their relatedness since frequently co-occurring entities are mostly related, but related entities might not co-occur frequently. 

Despite this success, we believe that there is potential for further improvement, if salience detection and entity relatedness would be customized for the entity aspect linking task. One issue is that relatedness is both unaware of the context and the aspect content. Extending entity relatedness measures to consider relatedness-in-context (similar to the prominence score) is likely to offer further improvements. Analogously, salience detection is currently trained for a linguistic purpose, that is unaware of the downstream task. We speculate that developing a new salience-like component that can identify which entities in the context are sufficiently central to be incorporated in the matching decision. Both a context-ware entity relatedness and a task-aware salience detector---once available---would also be useful for other downstream tasks.

Overall, we believe that our contribution to entity aspect-linking contributes to new knowledge-based information access systems. For once, it allows to construct knowledge graphs on the sub-entity level. While the edges are not typed, the fine-grained aspects model the roles that entities play in multi-way relations. Furthermore entity-aspect linking allows better information access for journalists, researchers, as well as any users who is seeking to understand fine-grained connections between entities through their aspects for open-domain information needs.