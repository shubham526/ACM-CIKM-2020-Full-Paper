\begin{abstract}
Entity linking tools had a significant impact on information retrieval. However, entity linking focuses only on the resolution and disambiguation of entities in a knowledge graph. In this work, we augment entity links with more fine-grained information clarifying which aspect of an entity is being referred to in this context. While many definitions of aspects can be leveraged in our formulation, we define Entity Aspect Linking as the task of associating an entity link with one of several known sub-aspects, such as Wikipedia sections.

Several pre-trained entity salience and relatedness detectors are available to predict salient entities in the text or the relatedness of two entities. However, these tools are not trained for our particular task. Salience detection is unaware of the entity's knowledge graph representation and entity relatedness is not considering the context. In this work, we study if and to what extent, such static, off-the-shelf tools can be utilized in the aspect linking task. Our results show that although a static measure of entity salience and relatedness from an off-the-shelf tool works on its own, a supervised combination of these indicators with lexical and semantic features based on the contexts of different sizes around the entity mention provides better results.


%Previous work \cite{nanni2018entity} has shown that a supervised combination of various text and entity features can correctly predict aspects in 70\% of the cases. In this work, we consider the salience of an entity in the aspect (that is, its centrality to the aspect) and its relatedness to other entities and obtain new state-of-the-art results on the task. Moreover, we study the effect of the frequency and relatedness of co-occurring entities with a given entity on the task.
\end{abstract}